%%%%%%%%%%%%%%%%%%%%%%%%%%%%%%%%%%%%%%%%%%%%%%%%%% Khai báo chung, vui lòng KHÔNG THAY ĐỔI phần này nếu không cần thiết
%%%%%%%%%%%%%%%%%%%% Các packages cơ bản
\usepackage[top=1.8cm, bottom=1.8cm, left=1.8cm, right=1.8cm]{geometry}
\usepackage[dvipsnames]{xcolor}
\usepackage[utf8]{vietnam}
\usepackage{amsmath,amsxtra,amssymb,latexsym, amscd,amsthm}
\usepackage{import}
\usepackage{graphicx,todo}
\usepackage{xstring}
\usepackage{mathrsfs}
\usepackage{tikz}
\usetikzlibrary{calc,intersections}
\usetikzlibrary{arrows}
\usepackage{tkz-base}
\usepackage{tkz-euclide}
\usepackage{tkz-tab}
\usepackage{import}
\usepackage{multicol}
\usepackage{listings}
\usepackage{lscape}
\usepackage{booktabs}
\usepackage{tabto}
\usepackage{caption}[=v3.0]
\usepackage[colorlinks,hyperindex,plainpages=false,unicode]{hyperref}
\newenvironment{tabs}[1]{\flushleft\TabPositions{#1}}{\endflushleft}
%%%%%%%%%%%%%%%%%%%%

%%%%%%%%%%%%%%%%%%%% Change Latex font
% \usepackage{mathpazo} 
%%%%%%%%%%%%%%%%%%%%

%%%%%%%%%%%%%%%%%%%% Sử dụng gói dethi.sty
\usepackage[baitap]{dethi}
%%%%%%%%%%%%%%%%%%%%

%%%%%%%%%%%%%%%%%%%% Bảng biến thiên
\usepackage{tkz-tab}
%%%%%%%%%%%%%%%%%%%%

%%%%%%%%%%%%%%%%%%%% K-maps
\input{kvmacros}
%%%%%%%%%%%%%%%%%%%%

%%%%%%%%%%%%%%%%%%%% K-maps_RosenBook
\usetikzlibrary{matrix,fit}
\tikzset{kmap/.style={ %% from Schrödinger's cat
    matrix of math nodes,nodes in empty cells,
    nodes={draw,minimum size=12mm,anchor=center},
    column 1/.style={nodes={draw=none,minimum size=0,text=blue,font=\normalsize}},
    row 1/.style={nodes={draw=none,minimum size=0,text=blue,font=\normalsize}},
    column sep=-.5*\pgflinewidth,
        row sep=-.5*\pgflinewidth
    }
}
%%%%%%%%%%%%%%%%%%%%

%%%%%%%%%%%%%%%%%%%% Bảng đáp án
\usepackage{longtable}
\usepackage{tabularx}
\parskip 3pt
\headsep=12pt
%%%%%%%%%%%%%%%%%%%%

%%%%%%%%%%%%%%%%%%%% Thiết lập
\trangcuoi{dbPage}
\hovaten{Họ và tên}         % Nếu không muốn có dòng này không gõ lệnh
% \tenlop{Tên lớp}          % Nếu không muốn có dòng này không gõ lệnh
\sobaodanh{Số báo danh}     % Nếu không muốn có dòng này không gõ lệnh
\khoanhd{\cboxv}{}
\daungoac{\cboxx}{}
\chuphuongan{\small\bfseries\Alph}
\mauchu{blue}
\PSNrandseed{\time}
\parindent 10pt
\graphicspath{{hinh-duan07/}{temp/}}
\duongkefalse
\usepackage{nonfloat}
\setlength{\baselineskip}{12truept}
\graphicspath{{hinh-maudethi/}}
% \nguondetrue
%%%%%%%%%%%%%%%%%%%%

%%%%%%%%%%%%%%%%%%%% Custom footnote
\usepackage[symbol]{footmisc}
\renewcommand{\thefootnote}{\fnsymbol{footnote}}
\makeatletter
\def\footnoterule{%
  \kern-3\p@
  \hbox{%
    %\tikz \fill[left color=blue!50!black, right color=white]
    \tikz \fill[left color=blue, right color=blue]
        (0,0) rectangle (6.1in,.5pt);%
  }%
  \kern 2.6\p@}
\makeatother
%%%%%%%%%%%%%%%%%%%%

%%%%%%%%%%%%%%%%%%%% Color packages
\usepackage[dvipsnames]{xcolor}
\usepackage{color}
\usepackage[framemethod=tikz]{mdframed}
% \definecolor{UMTBlue}{RGB}{50, 50, 225}
% \definecolor{UMTBlue}{RGB}{35, 25, 225}
% \definecolor{UMTRed}{RGB}{250, 25, 35}
\definecolor{UMTBlue}{RGB}{0, 0, 255}
\definecolor{UMTRed}{RGB}{255, 0, 0}
\mdfdefinestyle{theoremstyle}{%
  % linecolor=black,linewidth=1pt,%
  linecolor=UMTBlue,linewidth=1pt,%
  frametitlerule=true,%
  % frametitlebackgroundcolor=gray!20,%
  frametitlebackgroundcolor=UMTBlue,
  innertopmargin=\topskip,
}
%%%%%%%%%%%%%%%%%%%%

%%%%%%%%%%%%%%%%%%%% Watermark
\usepackage{draftwatermark}
\usepackage{tikz}
\SetWatermarkText{\tikz{\node[opacity=0.15]{\includegraphics{Resources/Logos/STAC.png}};}}
\SetWatermarkAngle{0}
\SetWatermarkScale{1.5}
%%%%%%%%%%%%%%%%%%%%


%%%%%%%%%%%%%%%%%%%% Footer và Header
\usepackage{fancyhdr}
\fancypagestyle{statementpage}{ % Trang đề bài
  \renewcommand{\headrulewidth}{0pt}
  \renewcommand{\footrulewidth}{0pt}
  \fancyfoot{}
  \fancyfoot[C]{\textcolor{black}{\textbf{Trang {\thepage}}}}
}

\fancypagestyle{solutionpage}{ % Trang lời giải
  \renewcommand{\headrulewidth}{0pt}
  \renewcommand{\footrulewidth}{0pt}
  \fancyfoot{}
  \fancyfoot[L]{\textcolor{black}{\textbf{CÂU LẠC BỘ HỌC THUẬT KHOA CÔNG NGHỆ UMT}} \\ \textcolor{black}{\text{UMT School of Technology Academic Club}}}
  \fancyfoot[R]{\textcolor{black}{\textbf{Trang {\thepage}}}}
}
%%%%%%%%%%%%%%%%%%%%
%%%%%%%%%%%%%%%%%%%% Chèn code C/C++
\definecolor{black}{cmyk}{0,0,0,1}             
\definecolor{white}{cmyk}{0,0,0,0}
\definecolor{darkgrey}{cmyk}{0,0,0,0.97}  
\definecolor{greyBackground}{cmyk}{0.0000, 0.0000, 0.0000, 0.0510}
\definecolor{greenComments}{cmyk}{
1.0000, 0.0000, 1.0000, 0.4980}
\definecolor{blueKeywords}{cmyk}{0.8222, 0.8222, 0.0000, 0.2941}
%\definecolor{pinkOtherKeywords}{cmyk}{0.1676, 0.4973, 0.0000, 0.2745}
\definecolor{pinkOtherKeywords}{cmyk}{0.0000, 0.2695, 0.7730, 0.4471}
\definecolor{brownFunctions}{cmyk}{0.0000, 0.1368, 0.3419, 0.5412}
\definecolor{orangeLibraries}{cmyk}{0.0000, 0.3871, 0.9677, 0.1490}
\definecolor{redStrings}{cmyk}{0.0000, 0.8712, 0.8712, 0.3608}
\lstset{
    language=C++,
    numbers=none,
    rulesep=10pt,
    %xleftmargin=12pt,
    %framexleftmargin=-2pt,
    %framexrightmargin=-5pt,
    showtabs=false,
    showspaces=false,
    showstringspaces=false,
    breaklines=true,
    breakatwhitespace=true,
    % backgroundcolor=\color{greyBackground},
    % rulecolor=\color{darkgrey},
    commentstyle=\color{greenComments},
    morecomment=[s][\color{greenComments}]{/*+}{*/},
    morecomment=[s][\color{greenComments}]{/*-}{*/},
    basicstyle=\ttfamily\color{black},
    stringstyle=\color{redStrings},
    % frame=trbl,
    frame=none,
    framesep=1pt,
    numbersep=7pt,
    belowcaptionskip=1\baselineskip,
    columns=fullflexible,
    captionpos=b,
    extendedchars=true,
    keepspaces=true, 
    stepnumber=5, 
    tabsize=4, 
    title=\lstname,
    % Keywords like string, int, false, true ...
    keywordstyle=\color{blue},
    morekeywords={partial, var, value, get, set},
    % Keywords like if/else, switch/case ...
    emphstyle=\color{pinkOtherKeywords},
    emph={if, else, return, throw, switch, case, while, do, while, using, break, continue, for},
    % Collection of your functions
    % emphstyle={[2]\color{brownFunctions}},
    % emph={[2]},
    % Collection of your libraries
    % emphstyle={[3]\color{orangeLibraries}},
    % emph={[3]iostream}
}
\newcolumntype{b}{X}
\newcolumntype{s}{>{\hsize=.75\hsize}X}
%%%%%%%%%%%%%%%%%%%%

%%%%%%%%%%%%%%%%%%%% Line spacing
\usepackage{setspace}
\renewcommand{\baselinestretch}{1.0} 
%%%%%%%%%%%%%%%%%%%%

%%%%%%%%%%%%%%%%%%%% Check if a page is odd or even
\usepackage{scrextend}
%%%%%%%%%%%%%%%%%%%%

%%%%%%%%%%%%%%%%%%%% Custom column style
\newcolumntype{C}[1]{>{\centering\arraybackslash}p{#1}}
%%%%%%%%%%%%%%%%%%%%

%%%%%%%%%%%%%%%%%%%% Custom caption style
\captionsetup[table]{labelfont=bf, labelsep=period, justification=centering}
\captionsetup[figure]{labelfont=bf, labelsep=period, justification=centering}
%%%%%%%%%%%%%%%%%%%%

%%%%%%%%%%%%%%%%%%%% Custom matrix
\makeatletter
\renewcommand*\env@matrix[2][1.0]{%
\edef\arraystretch{#1}%
\hskip -\arraycolsep
\let\@ifnextchar\new@ifnextchar
\array{#2}}
\makeatother
%%%%%%%%%%%%%%%%%%%%

%%%%%%%%%%%%%%%%%%%% Substack alignment
\makeatletter
\newcommand{\subalign}[1]{%
  \vcenter{%
    \Let@ \restore@math@cr \default@tag
    \baselineskip\fontdimen10 \scriptfont\tw@
    \advance\baselineskip\fontdimen12 \scriptfont\tw@
    \lineskip\thr@@\fontdimen8 \scriptfont\thr@@
    \lineskiplimit\lineskip
    \ialign{\hfil$\m@th\scriptstyle##$&$\m@th\scriptstyle{}##$\hfil\crcr
      #1\crcr
    }%
  }%
}
\makeatother
%%%%%%%%%%%%%%%%%%%% 

%%%%%%%%%%%%%%%%%%%% Diagonal box
\usepackage{diagbox}
%%%%%%%%%%%%%%%%%%%%

%%%%%%%%%%%%%%%%%%%% Customize enumerate, https://ctan.math.illinois.edu/macros/latex/contrib/tasks/tasks-manual.pdf
\usepackage{enumitem}
\usepackage{tasks}
%%%%%%%%%%%%%%%%%%%%

%%%%%%%%%%%%%%%%%%%% Exercise box
\usepackage{setspace}
\usepackage{thmtools}
\usepackage{framed}
\usepackage{tcolorbox}
\usepackage{environ}
\tcbuselibrary{theorems,skins,breakable}

\usepackage{xpatch}
\makeatletter
\xpatchcmd{\@thmr}{:}{.}{}{}
\makeatother

\newtcbtheorem[auto counter]{mytheorem}{Theorem}
{
    enhanced,
    frame hidden,
    titlerule=0mm,
    toptitle=1mm,
    bottomtitle=1mm,
    fonttitle=\bfseries,
    coltitle=black,
    colbacktitle=cyan!20!white,
    colback=cyan!10!white,
    separator sign={. },
}{thm}

\NewDocumentCommand{\thm}{m+m}{
    \begin{mytheorem}{#1}{}
        #2
    \end{mytheorem}
}

\NewDocumentCommand{\thmr}{mm+m}{
    \begin{mytheorem}{#1}{#2}
        #3
    \end{mytheorem}
}
%%%%%%%%%%%%%%%%%%%%

%%%%%%%%%%%%%%%%%%%%%%%%%%%%%%%%%%%%%%%%%%%%%%%%%% Kết thúc khai báo chung