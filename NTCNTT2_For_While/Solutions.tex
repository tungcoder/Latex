%%%%%%%%%%%%%%%%%%%% Template
\lstset{
  breaklines=true,
  breakatwhitespace=true,
  keepspaces=true,
  basicstyle=\fontfamily{zi4}\selectfont\large,
  columns=fullflexible,
  inputencoding=utf8,
  extendedchars=true
}

\begin{center}
    \textbf{\textcolor{black}{\large\LARGE NỀN TẢNG CÔNG NGHỆ THÔNG TIN 2}}

    \textbf{\large CÁCH SỬ DỤNG VÒNG LẶP FOR, WHILE}

    \large\LaTeX\, bởi Ngô Hoàng Tùng
\end{center}

% -----------------------------------------------------

\subsection*{\textbf{I. Mục tiêu}}
\begin{enumerate}[label=\alph*.]
    \item Cấu trúc lặp FOR, WHILE
    \item Cách sử dụng vòng lặp FOR, WHILE
    \item Cách sử dụng vòng lặp lồng nhau
    \item Cách sử dụng vòng lặp với mảng 1 chiều
    \item Cách sử dụng vòng lặp với mảng 2 chiều
\end{enumerate}
\subsection*{\textbf{II. Nội dung}}
\begin{enumerate}[label=\alph*.]
    \item Cấu trúc lặp FOR, WHILE
    \begin{itemize}
        \item Cấu trúc lặp FOR: Dùng để lặp một đoạn mã với số lần lặp xác định.
        \item Cấu trúc lặp WHILE: Dùng để lặp một đoạn mã khi điều kiện là đúng.
    \end{itemize}

    \item Cách sử dụng vòng lặp FOR, WHILE
    \begin{itemize}
        \item Vòng lặp FOR:

        Cú pháp:
        \begin{lstlisting}
        for i in range(start, end, step):
            # Do something
        \end{lstlisting}
        \vspace{-3.5em}
        Hoặc:
        \begin{lstlisting}
        for i in range(start, end, step):
            # Do something
        \end{lstlisting}
        \vspace{-3.5em}
        Giải thích:
        \begin{itemize}
            \item \texttt{start}: giá trị bắt đầu (mặc định là 0)
            \item \texttt{end}: giá trị kết thúc (không bao gồm giá trị này)
            \item \texttt{step}: bước nhảy (mặc định là 1)
            \item \texttt{range(end)}: lặp từ 0 đến end-1
            \item \texttt{range(start, end)}: lặp từ start đến end-1
            \item \texttt{range(start, end, step)}: lặp từ start đến end-1 với bước nhảy là step
        \end{itemize}

        Ví dụ:
        \begin{lstlisting}
        for i in range(5):
            print(i)  # In ra 0, 1, 2, 3, 4
        \end{lstlisting}
        \vspace{-4.5em}
        -- range(5) tương đương với range(0, 5, 1) hoặc range (0, 5).
    \end{itemize}

    \begin{itemize}
        \item Vòng lặp WHILE:

        Cú pháp:
        \begin{lstlisting}
        while condition:
            # Do something
        \end{lstlisting}
        \vspace{-3.5em}
        Giải thích:
        \begin{itemize}
            \item \texttt{condition}: điều kiện để tiếp tục lặp (nếu điều kiện là đúng thì lặp)
        \end{itemize}

        Ví dụ 1:
        \begin{lstlisting}
        i = 0
        while i < 5:
            print(i)  # In ra 0, 1, 2, 3, 4
            i += 1
        \end{lstlisting}
        \vspace{-4.5em}
        -- Vòng lặp while sẽ tiếp tục lặp cho đến khi điều kiện là sai.\\
        Ví dụ 2:
        \begin{lstlisting}
            while True:
                print("Hello, World!")
        \end{lstlisting}
        \vspace{-4.5em}
        -- Vòng lặp while sẽ lặp vô hạn vì điều kiện luôn đúng. Để dừng vòng lặp, bạn có thể sử dụng lệnh \texttt{break}.
        \begin{lstlisting}
            while True:
                print("Hello, World!")
                break
        \end{lstlisting}
    \end{itemize}
    \vspace{-6em}
    \item Cách sử dụng vòng lặp lồng nhau
    \begin{itemize}
        \item Vòng lặp lồng nhau là khi bạn đặt một vòng lặp bên trong một vòng lặp khác. Điều này cho phép bạn lặp qua nhiều cấp độ dữ liệu.
        \item Cú pháp:
        \begin{lstlisting}
        for i in range(n):
            for j in range(m):
                # Do something
        \end{lstlisting}
        \vspace{-4.5em}
        Hoặc:
        \begin{lstlisting}
        while condition1:
            while condition2:
                # Do something
        \end{lstlisting}
        \vspace{-4.5em}
        Giải thích:
        \begin{itemize}
            \item Vòng lặp ngoài sẽ lặp qua các giá trị từ 0 đến n-1.
            \item Vòng lặp trong sẽ lặp qua các giá trị từ 0 đến m-1 cho mỗi giá trị của vòng lặp ngoài.
        \end{itemize}
        Ví dụ 1:
        \begin{lstlisting}
        for i in range(3):
            for j in range(2):
                print(f"i: {i}, j: {j}")
        \end{lstlisting}
        \vspace{-4.5em}
        -- Kết quả sẽ là:\\
        i: 0, j: 0\\
        i: 0, j: 1\\
        i: 1, j: 0\\
        i: 1, j: 1\\
        i: 2, j: 0\\
        i: 2, j: 1\\
        Ví dụ 2:
        \begin{lstlisting}
        i = 0
        while i < 3:
            j = 0
            while j < 2:
                print(f"i: {i}, j: {j}")
                j += 1
            i += 1
        \end{lstlisting}
        \vspace{-4.5em}
        -- Kết quả như ví dụ 1.
    \end{itemize}
    \item Cách sử dụng vòng lặp với mảng 1 chiều
    \begin{itemize}
        \item Khi bạn có một mảng 1 chiều, bạn có thể sử dụng vòng lặp lồng nhau để truy cập từng phần tử của mảng.
        \item Cú pháp:
        \begin{lstlisting}
        for i in range(len(array)):
            # Do something with array[i]
        \end{lstlisting}
        \vspace{-4.5em}
        Hoặc:
        \begin{lstlisting}
        while condition:
            # Do something with array[i]
        \end{lstlisting}
        \vspace{-4.5em}
        Giải thích:
        \begin{itemize}
            \item Sử dụng vòng lặp for hoặc while để lặp qua các chỉ số của mảng.
            \item Truy cập từng phần tử của mảng bằng cách sử dụng chỉ số.
        \end{itemize}
        Ví dụ:
        \begin{lstlisting}
        array = [1, 2, 3, 4, 5]
        for i in range(len(array)):
            print(array[i], end=' ')
        \end{lstlisting}
        \vspace{-4.5em}
        -- Kết quả sẽ là: 1 2 3 4 5\\
        Hoặc:
        \begin{lstlisting}
        array = [1, 2, 3, 4, 5]
        i = 0
        while i < len(array):
            print(array[i], end=' ')  
            i += 1
        \end{lstlisting}
        \vspace{-4.5em}
        -- Kết quả cũng sẽ là: 1 2 3 4 5
    \end{itemize}
    \item Cách sử dụng vòng lặp với mảng 2 chiều
    \begin{itemize}
        \item Khi bạn có một mảng 2 chiều, bạn có thể sử dụng vòng lặp lồng nhau để truy cập từng phần tử của mảng.
        \item Cú pháp:
        \begin{lstlisting}
        for i in range(len(array)):
            for j in range(len(array[i])):
                # Do something with array[i][j]
        \end{lstlisting}
        \vspace{-4.5em}
        Hoặc:
        \begin{lstlisting}
        while condition1:
            while condition2:
                # Do something with array[i][j]
        \end{lstlisting}
        \vspace{-4.5em}
        Giải thích:
        \begin{itemize}
            \item Sử dụng vòng lặp for hoặc while để lặp qua các chỉ số của mảng 2 chiều.
            \item Truy cập từng phần tử của mảng bằng cách sử dụng chỉ số hàng và cột.
        \end{itemize}
        Ví dụ:
        \begin{lstlisting}
        array = [[1, 2, 3], [4, 5, 6], [7, 8, 9]]
        for i in range(len(array)):
            for j in range(len(array[i])):
                print(array[i][j], end=' ')
        \end{lstlisting}
        \vspace{-4.5em}
        -- Kết quả sẽ là: 1 2 3 4 5 6 7 8 9\\
        Hoặc:
        \begin{lstlisting}
        array = [[1, 2, 3], [4, 5, 6], [7, 8, 9]]
        i = 0
        while i < len(array):
            j = 0
            while j < len(array[i]):
                print(array[i][j], end=' ')
                j += 1
            i += 1
        \end{lstlisting}
        \vspace{-4.5em}
        -- Kết quả cũng sẽ là: 1 2 3 4 5 6 7 8 9
    \end{itemize}
\end{enumerate}
\subsection*{\textbf{III. Bài tập}}
\subsection*{\textbf{Bài 1: }}
\textcolor{black}{Cho mảng 1 chiều gồm $n$ số nguyên dương, hãy tính tổng các số trong mảng.}\\
\textcolor{black}{\textbf{Input: }}\\
\textcolor{black}{Dòng đầu chứa số nguyên dương duy nhất $n$ (0 < $n$ < 1000) là số lượng phần tử của mảng.}\\
\textcolor{black}{Dòng thứ hai chứa $n$ số nguyên là các phần tử của mảng $a_{0}, a_{1}, \ldots, a_{n-1}$ ($-10^9 \leq a_{i} \leq 10^9$).}\\
\textcolor{black}{\textbf{Output: }}\\
\textcolor{black}{In ra một số nguyên là tổng các phần tử của mảng.}\\
\begin{table}[h!]
\centering
\begin{tabularx}{0.8\textwidth}{|X|X|}
\hline
\textbf{Input} & \textbf{Output} \\
\hline
5 & 19 \\
9 \quad -5 \quad 20 \quad -10 \quad 5 & \\
\hline
\end{tabularx}
\end{table}\\

\subsection*{\textbf{Bài 2: }}
\textcolor{black}{Cho $n$ và $m$ đều là số nguyên dương. Hãy xuất 1 ma trận $a_{i, j}$ có kích thước $n$ x $m$ với $a_{i, j}$ tăng dần từ 1 tới $n$ x $m$ và theo đường ZigZag.}\\
\textcolor{black}{\textbf{Input: }}\\
\textcolor{black}{Dòng đầu chứa hai số nguyên dương $n$, $m$ (1 $\leq$ $n, m$ $\leq$ 1000) là kích thước của ma trận.}\\
\textcolor{black}{\textbf{Output: }}\\
\textcolor{black}{In ra ma trận $a_{i, j}$ theo định dạng như ví dụ dưới đây.}\\

\begin{table}[h!]
\centering
\begin{tabularx}{0.8\textwidth}{|X|X|}
\hline
\textbf{Input} & \textbf{Output} \\
\hline
3 \quad 4 & 1\quad 2\quad 3\quad 4 \\
& 8\quad 7 \quad 6\quad 5 \\
& 9\quad 10\quad 11\quad 12 \\
\hline
\end{tabularx}
\end{table}

\subsection*{\textbf{Bài 3: }}
\textcolor{black}{Cho $n$ là số nguyên dương. Hãy xuất ra dãy số nguyên tố bé hơn hoặc bằng $n$.}\\
\textcolor{black}{\textbf{Input: }}\\
\textcolor{black}{Dòng đầu chứa số nguyên dương duy nhất $n$ (1 $\leq n \leq$ 100) là số nguyên cần kiểm tra.}\\
\textcolor{black}{\textbf{Output: }}\\
\textcolor{black}{In ra các số nguyên tố bé hơn hoặc bằng $n$.}\\
\begin{table}[h!]
\centering
\begin{tabularx}{0.8\textwidth}{|X|X|}
\hline
\textbf{Input} & \textbf{Output} \\
\hline
10 & 2\quad 3\quad 5\quad 7 \\
\hline
\end{tabularx}
\end{table}

\subsection*{\textbf{Bài 4:}}
\textcolor{black}{Cho $n, m$ là số nguyên dương và 1 ma trận $n$ x $m$. Hãy sắp xếp các phần tử trong ma trận theo thứ tự tăng dần theo cả cột và dòng.}\\
\textcolor{black}{\textbf{Input: }}\\
\textcolor{black}{Dòng đầu chứa hai số nguyên dương $n$, $m$ (1 $\leq$ $n, m$ $\leq$ 100) là kích thước của ma trận.}\\
\textcolor{black}{Dòng thứ hai chứa $n$ x $m$ số nguyên là các phần tử của ma trận.}\\
\textcolor{black}{\textbf{Output: }}\\
\textcolor{black}{In ra các phần tử của ma trận đã sắp xếp theo thứ tự tăng dần.}\\
\begin{table}[h!]
\centering
\begin{tabularx}{0.8\textwidth}{|X|X|}
\hline
\textbf{Input} & \textbf{Output} \\
\hline
3 \quad 3 & 1\quad 4\quad 6\\
5 \quad 1 \quad 7 & 2\quad 5\quad 7\\
9 \quad 11 \quad 3 & 3\quad 9\quad 11\\
2 \quad 6 \quad 4 & \\
\hline
\end{tabularx}
\end{table}

\subsection*{\textbf{IV. Chia sẻ kinh nghiệm}}
-- Mình khuyên các bạn nên sử dụng vòng lặp for khi bạn biết trước số lần lặp, và sử dụng vòng lặp while khi bạn không biết trước số lần lặp.\\
-- For sẽ tốt hơn while trong trường hợp bạn cần lặp qua một dãy số hoặc một mảng, vì nó giúp bạn dễ dàng kiểm soát chỉ số và điều kiện lặp. Còn while sẽ hữu ích hơn khi bạn cần lặp cho đến khi một điều kiện nào đó được thỏa mãn.\\
-- Khi sử dụng vòng lặp lồng nhau, hãy chú ý đến hiệu suất của chương trình, vì vòng lặp lồng nhau có thể làm tăng độ phức tạp của thuật toán.\\
-- Khi làm việc với mảng, hãy chắc chắn rằng bạn đã hiểu rõ về kích thước và chỉ số của mảng để tránh lỗi truy cập ngoài phạm vi.\\
-- Hãy luôn kiểm tra điều kiện dừng của vòng lặp để tránh vòng lặp vô hạn, điều này có thể làm treo chương trình của bạn.\\
-- Sử dụng lệnh \texttt{break} và \texttt{continue} để điều khiển luồng của vòng lặp, giúp bạn có thể thoát khỏi vòng lặp hoặc bỏ qua một lần lặp nếu cần thiết.\\
-- Hãy sử dụng các công cụ gỡ lỗi để theo dõi giá trị của biến trong vòng lặp, điều này sẽ giúp bạn dễ dàng phát hiện lỗi hơn.\\
-- Cuối cùng, hãy thực hành nhiều để làm quen với các cấu trúc lặp, vì chúng là một phần quan trọng trong lập trình và sẽ giúp bạn giải quyết nhiều bài toán khác nhau.
